\documentclass{article}
\title{Kryptografia - Projekt}
\author{Piotr Berezowski, Wojciech Zaniewski}

\usepackage{polski}
\usepackage[utf8]{inputenc}
\usepackage{enumerate}
\usepackage{amsmath}
\usepackage{amsfonts}
\usepackage{hyperref}
\usepackage{cleveref}
\usepackage{cases}
\usepackage{mathtools}
\usepackage{float}
\usepackage{graphicx}
\usepackage{caption}
\usepackage{subcaption}
\usepackage[ruled,vlined,linesnumbered,longend]{algorithm2e}
\graphicspath{ {./src/} }

\newenvironment{pseudokod}[1][htb]{
	\renewcommand{\algorithmcfname}{}
	\begin{algorithm}[#1]%
	}{
\end{algorithm}
}

\begin{document}
	\maketitle
	\pagenumbering{gobble}
	\newpage
    \pagenumbering{arabic}


    \section{Temat projektu}

    Prepare a mobile app(s) which implements Feige-Fiat-Shamir identification protocol. Such an app can be used then as user's digital ID.
    
    \section{Protokół Feige-Fiat-Shamir}


    \section{Opis projektu}

    Przy rejestracji użytkownik oblicza klucz publiczny i wysyła do serwera razem z loginem / imieniem. Serwer zapisuje w bazie klucz publiczny 
    użytkownika.

    Podczas próby logowania mamy k razy weryfikacje użytkownika, jeśli wszystkie próby się powiodły, to zalogowany.

    
\end{document}